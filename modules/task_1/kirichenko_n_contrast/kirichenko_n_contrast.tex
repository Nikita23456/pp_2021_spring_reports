\documentclass{report}

\usepackage[T2A]{fontenc}
\usepackage[utf8]{luainputenc}
\usepackage[english, russian]{babel}
\usepackage[pdftex]{hyperref}
\usepackage[14pt]{extsizes}
\usepackage{listings}
\usepackage{color}
\usepackage{geometry}
\usepackage{enumitem}
\usepackage{multirow}
\usepackage{graphicx}
\usepackage{indentfirst}
\usepackage{amsmath}

\geometry{a4paper,top=2cm,bottom=3cm,left=2cm,right=1.5cm}
\setlength{\parskip}{0.5cm}
\setlist{nolistsep, itemsep=0.3cm,parsep=0pt}

\lstset{language=C++,
		basicstyle=\footnotesize,
		keywordstyle=\color{blue}\ttfamily,
		stringstyle=\color{red}\ttfamily,
		commentstyle=\color{green}\ttfamily,
		morecomment=[l][\color{magenta}]{\#}, 
		tabsize=4,
		breaklines=true,
  		breakatwhitespace=true,
  		title=\lstname,       
}

\begin{document}

\begin{titlepage}
\begin{center}
Министерство образования и науки Российской Федерации
\end{center}
\begin{center}
Федеральное государственное автономное образовательное учреждение высшего образования \\
Национальный исследовательский Нижегородский государственный университет им. Н.И. Лобачевского(ННГУ)
\end{center}
\begin{center}
Институт информационных технологий, математики и механики
\end{center}
\begin{center}
Направление подготовки: «Фундаментальная информатика и информационные технологии»
\end{center}

\vspace{2em}

\begin{center}
\textbf{\LargeОтчет по лабораторной работе} 
\end{center}
\begin{center}
\textbf{\Large«Повышение контраста полутонового изображения посредством линейной растяжки гистограммы»} \\
\end{center}

\vspace{4em}

\newbox{\lbox}
\savebox{\lbox}{\hbox{text}}
\newlength{\maxl}
\setlength{\maxl}{\wd\lbox}
\hfill\parbox{7cm}{
\textbf{Выполнил:} \\ студент группы 381806-3 \\ Кириченко Н. А.\\
\\
\textbf{Проверил:}\\ доцент кафедры МОСТ, \\ кандидат технических наук \\ Сысоев А. В.\\
}
\vspace{\fill}
\begin{center} Нижний Новгород \\ 2021 \end{center}
\end{titlepage}

\setcounter{page}{2}

% Оглавление
\tableofcontents
\newpage

\section*{Введение}
\addcontentsline{toc}{section}{Введение}
Предварительная обработка изображения -процесс улучшения качества изображения, ставящий целью получение на основе оригинала максимально точного и адаптированного для автоматического анализа изображения.
\par Контраст представляет собой характеристику того, насколько большой разброс имеют цвета пикселей изображения. Чем больший разброс имеют значения цветов пикселей, тем больший контраст имеет изображение.
\par Слабый контраст - наиболее распространенный дефект фотографических, телевизионных и сканерных изображений, обусловленный ограниченностью диапазона воспроизводимых яркостей. Под контрастом обычно понимают разность максимального и минимального значений яркости. Увеличение диапазона интенсивности изображения существенно улучшает качество изображения за счет усиления контраста. Для этого используют три метода – линейная растяжка, нормализация и выравнивание гистограмм. Все эти методы направлены на расширение диапазона интенсивности до максимального интервала от 0 до 255.
\newpage

\section*{Постановка задачи}
\addcontentsline{toc}{section}{Постановка задачи}
\par Цель работы – разработка последовательной и параллельных реализаций алгоритма повышения контраста полутонового изображения посредством линейной растяжки гистограммы.
\par Линейная растяжка сводится к присваиванию новых значений интенсивности каждому пикселю изображения. Если интенсивности исходного изображения изменялись в диапазоне от $i_{1}$ до $i_{2}$, тогда необходимо линейно "растянуть" указанный диапазон так, чтобы значения изменялись от 0 до 255. Для этого достаточно пересчитать старые значения интенсивности $f_{xy}$ для всех пикселей $(x,y)$ согласно формуле $g_{xy} = a \cdot f_{xy} + b$, где коэффициенты a,b просто вычисляются, исходя из того, что граница $i_{1}$ должна перейти в 0, а $i_{2}$ – в 255.
\par Нужно реализовать последовательную версию программы, а так сделать параллельную реализацию используя TBB и OpenMP.
\par Для проверки корректности работы алгоритмов требуется использовать Google C++ Testing Framework.
\newpage

\section*{Описание алгоритма}
\addcontentsline{toc}{section}{Описание алгоритма}
Алгоритм повышения контраста полутонового изображения посредством линейной растяжки гистограммы заключается в следующем:
\begin{enumerate}
\item Найти $min$ и $max$.
\item Пересчитать старые значения интенсивности для всех пикселей по формуле $f^{-1}(y)$ = $((y - y_{min}) * 255) / ({{y_{max} - y_{min}})}$.
\end{enumerate}
\newpage

\section*{Описание схемы распараллеливания}
\addcontentsline{toc}{section}{Описание схемы распараллеливания}
\par В данном алгоритме, распараллеливание происходит путем разбиения данных на части и передачи их на потоки.
\par Первым шагом идет разбиение изображения на части, в которых находятся локальные   $min$ и $max$, а потом находятся глобальные $min$ и $max$.
\par Заключающим шагом является пересчитывание старых значений интенсивности каждым потоком по формуле $f^{-1}(y)$.

\newpage

\section*{Описание программной реализации}
\addcontentsline{toc}{section}{Описание программной реализации}
\par Для  последовательного алгоритма следует вызвать функцию:
\begin{lstlisting}
Result Contrast(const Result& rm);
\end{lstlisting}
\par Входным параметром передается изображение
\par Для изменения контраста полутонового изображения с использованием нескольких потоков следует вызвать функции:
\begin{lstlisting}
Result Contrastomp(const Result& rm);
Result Contrasttbb(const Result& rm);
Result Contraststd(const Result& rm);
\end{lstlisting}
\par Входным параметром все так же является изображение.
\newpage

\section*{Результаты экспериментов}
\addcontentsline{toc}{section}{Результаты экспериментов}
Вычислительные эксперименты проводились на оборудовании со следующей аппаратной конфигурацией:
\begin{itemize}
\item Процессор: Intel(R) Core(TM) i7-2600K CPU @ 3.40 GHz, 3400 МГц, ядер 4;
\item Оперативная память: 16 ГБ (DDR3), 1200 MHz;
\item ОС: Microsoft Windows 10 pro 
\end{itemize}
\par Изображение 10000*10000
\\
\begin{table}[!h]
\centering
\begin{tabular}{lllll}
Версия & Время (сек) & Ускорение  \\
Seq   & 9.13 & -  \\
OpenMP  & 4.38 & 2.084  \\
TBB  & 2.59 & 3.525  \\
Std  & 3.15 & 2.898  \\
\end{tabular}
\caption{Результаты экспериментов}
\end{table}

\newpage

\section*{Заключение}
\addcontentsline{toc}{section}{Заключение}
\par В ходе лабораторной работы, мы разработали последовательную и параллельные реализации алгоритма повышения контраста полутонового изображения посредством линейной растяжки гистограммы с использованием  OpenMP, TBB и std::thread.
\par Для подтверждения корректности работы программы разработан и доведен до успешного
выполнения набор тестов с использованием библиотеки модульного тестирования Google C++
Testing Framework.
\par По данным экспериментов удалось сравнить время работы параллельных и последовательного
алгоритмов. Выявлено, что параллельный алгоритм повышения контраста полутонового изображения посредством линейной растяжки гистограммы показывает высокую эффективность на достаточно большом объеме данных.
\newpage

\begin{thebibliography}{1}
\addcontentsline{toc}{section}{Литература}
\bibitem{Gorodetsky} Турлапов В. Е. «Обработка изображений. Часть 1».
\\URL:\url {http://www.graph.unn.ru/rus/materials/CG/CG03_ImageProcessing.pdf} (дата обращения: 22.05.2021)
\bibitem{} «Параллельное программирование на OpenMP»
\\URL:\url {http://ccfit.nsu.ru/arom/data/openmp.pdf} (дата обращения: 22.05.2021)
\bibitem{Sidnev} Сиднев А.А., Мееров И.Б., Сысоев А.В. «Разработка параллельных программ в системах с общей памятью с использованием библиотеки Intel Threading Building Blocks (TBB)».
\\URL:\url {http://hpc-education.ru/files/lectures/meerov/ppt06.pdf} (дата обращения: 22.05.2021)
\bibitem{} «Параллельное программирование. С++. Thread Support Library. Atomic Operations Library.»
\\URL:\url {https://ppt-online.org/184002} (дата обращения: 22.05.2021)
\end{thebibliography}
\newpage

\section*{Приложение}
\addcontentsline{toc}{section}{Приложение}
В данном разделе находится листинг всего кода, написанного в рамках лабораторной работы.
\begin{lstlisting}
// contrast.h
#include <iostream>
#include <cstdint>
#include <random>
#include <vector>
#include <cassert>
#include <ctime>
#include <algorithm>
#include <limits>
#include <utility>

using Result = std::vector<int>;

Result RandomI(int l, int k);
Result Contrast(const Result& rm);
Result Contrastomp(const Result& rm);
Result Contrasttbb(const Result& rm);
Result Contraststd(const Result& rm);
\end{lstlisting}
\begin{lstlisting}
//contrast.cpp
#include "../../../modules/task_4/kirichenko_n_contrast/contrast.h"
#include "../../../3rdparty/unapproved/unapproved.h"

Result RandomI(int l, int k) {
    if (l <= 0 || k <= 0)
        throw std::runtime_error("Incorrect data!");
    static std::mt19937 random(time(0));
    Result res(l * k);
    for (int i = 0; i < l; i++) {
        for (int j = 0; j < k; j++) {
            res[i * l + j] = random() % 256;
        }
    }
    return res;
}

Result Contrast(const Result& rm) {
    Result outcome(rm.size());
    int max = 0;
    int min = 255;
    for (int i = 0; i < static_cast<int>(rm.size()); i++) {
        if (min > rm[i])
            min = rm[i];
        if (max < rm[i])
            max = rm[i];
    }
    for (int i = 0; i < static_cast<int>(rm.size()); i++) {
        outcome[i] = (rm[i] - min) * (255 / (max - min));
    }
    return outcome;
}

Result Contrastomp(const Result& rm) {
    Result outcome(rm.size());
    int const num_threads = 4;
    omp_set_num_threads(num_threads);
    int max_n[num_threads], min_n[num_threads];
    int max = 0, min = 255;
#pragma omp parallel
    {
       int nT = omp_get_thread_num();
        min_n[nT] = 255;
        max_n[nT] = 0;
#pragma omp for
        for (int i = 0; i < static_cast<int>(rm.size()); i++) {
            if (min_n[nT] > rm[i])
                min_n[nT] = rm[i];
            if (max_n[nT] < rm[i])
                max_n[nT] = rm[i];
        }
    }
#pragma omp for
    for (int i = 0; i < num_threads; i++) {
        if (min_n[i] < min)
            min = min_n[i];
        if (max_n[i] > max)
            max = max_n[i];
    }
#pragma omp for
    for (int i = 0; i < static_cast<int>(rm.size()); i++) {
        outcome[i] = ((rm[i] - min) * 255) / (max - min);
    }
    return outcome;
}

Result Contrasttbb(const Result& rm) {
    Result outcome(rm.size());
    int min = 255, max = 0;

    int const numThreads = 4;

    tbb::task_scheduler_init init(numThreads);

    size_t gs = ((rm.size() / 1000) > 1 ? (rm.size() / 1000) : 1);
    tbb::parallel_for(tbb::blocked_range<size_t>(0, rm.size(), gs),
        [&](const tbb::blocked_range<size_t> &r) {
            for (size_t i = r.begin(); i != r.end(); ++i) {
                if (rm[i] > max) {
                    max = rm[i];
                }
                if (rm[i] < min) {
                    min = rm[i];
                }
            }
        }, tbb::simple_partitioner());

    tbb::parallel_for(0, static_cast<int>(rm.size()), [&](const int i) {
        outcome[i] = ((rm[i] - min) * 255) / (max - min);
        });
    return outcome;
}

Result Contraststd(const Result& rm) {
    Result outcome(rm.size());
    int min = 255, max = 0;

    int const numThreads = 4;

    int delta = static_cast<int>(rm.size()) / numThreads;
    std::vector<int> limits(numThreads + 1);
    for (int i = 0; i < numThreads; i++) {
        limits[i] = i * delta;
    }

    limits[numThreads] = static_cast<int>(rm.size());

    std::vector <std::pair<int, int>> min_max_arr(numThreads);

    std::vector<std::thread> threads;

    for (int i = 0; i < numThreads; i++) {
        std::thread thr([&rm, &limits, &min_max_arr](int id_thread) -> void {
            std::pair<int, int> result(255, 0);
            for (int i = limits[id_thread]; i < limits[id_thread + 1]; i++) {
                if (rm[i] > result.second) {
                    result.second = rm[i];
                }
                if (rm[i] < result.first) {
                    result.first = rm[i];
                }
            }
            min_max_arr[id_thread] = result;
            }, i);
        threads.emplace_back(std::move(thr));
    }
    for (int i = 0; i < numThreads; i++) {
        threads[i].join();
    }

    for (auto& temp : min_max_arr) {
        if (temp.second > max) {
            max = temp.second;
        }
        if (temp.first < min) {
            min = temp.first;
        }
    }

    std::vector<std::thread> threads1;
    for (int i = 0; i < numThreads; i++) {
        std::thread thr([&](const int i) {
            for (int j = limits[i]; j < limits[i + 1]; j++)
                outcome[j] = ((rm[j] - min) * 255) / (max - min);
            }, i);
        threads1.emplace_back(std::move(thr));
    }
    for (int i = 0; i < numThreads; i++) {
        threads1[i].join();
    }
    return outcome;
}
\end{lstlisting}

\end{document}