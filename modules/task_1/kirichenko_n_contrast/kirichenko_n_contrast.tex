\documentclass{report}

\usepackage[T2A]{fontenc}
\usepackage[utf8]{luainputenc}
\usepackage[english, russian]{babel}
\usepackage[pdftex]{hyperref}
\usepackage[14pt]{extsizes}
\usepackage{listings}
\usepackage{color}
\usepackage{geometry}
\usepackage{enumitem}
\usepackage{multirow}
\usepackage{graphicx}
\usepackage{indentfirst}
\usepackage{amsmath}

\geometry{a4paper,top=2cm,bottom=3cm,left=2cm,right=1.5cm}
\setlength{\parskip}{0.5cm}
\setlist{nolistsep, itemsep=0.3cm,parsep=0pt}

\lstset{language=C++,
		basicstyle=\footnotesize,
		keywordstyle=\color{blue}\ttfamily,
		stringstyle=\color{red}\ttfamily,
		commentstyle=\color{green}\ttfamily,
		morecomment=[l][\color{magenta}]{\#}, 
		tabsize=4,
		breaklines=true,
  		breakatwhitespace=true,
  		title=\lstname,       
}

\begin{document}

\begin{titlepage}
\begin{center}
Министерство образования и науки Российской Федерации
\end{center}
\begin{center}
Федеральное государственное автономное образовательное учреждение высшего образования \\
Национальный исследовательский Нижегородский государственный университет им. Н.И. Лобачевского(ННГУ)
\end{center}
\begin{center}
Институт информационных технологий, математики и механики
\end{center}
\begin{center}
Направление подготовки: «Фундаментальная информатика и информационные технологии»
\end{center}

\vspace{2em}

\begin{center}
\textbf{\LargeОтчет по лабораторной работе} 
\end{center}
\begin{center}
\textbf{\Large«Повышение контраста полутонового изображения посредством линейной растяжки гистограммы»} \\
\end{center}

\vspace{4em}

\newbox{\lbox}
\savebox{\lbox}{\hbox{text}}
\newlength{\maxl}
\setlength{\maxl}{\wd\lbox}
\hfill\parbox{7cm}{
\textbf{Выполнил:} \\ студент группы 381806-3 \\ Кириченко Н. А.\\
\\
\textbf{Проверил:}\\ доцент кафедры МОСТ, \\ кандидат технических наук \\ Сысоев А. В.\\
}
\vspace{\fill}
\begin{center} Нижний Новгород \\ 2021 \end{center}
\end{titlepage}

\setcounter{page}{2}

% Оглавление
\tableofcontents
\newpage

\section*{Введение}
\addcontentsline{toc}{section}{Введение}
Предварительная обработка изображения -процесс улучшения качества изображения, ставящий целью получение на основе оригинала максимально точного и адаптированного для автоматического анализа изображения.
\par Контраст представляет собой характеристику того, насколько большой разброс имеют цвета пикселей изображения. Чем больший разброс имеют значения цветов пикселей, тем больший контраст имеет изображение.
\par Слабый контраст - наиболее распространенный дефект фотографических, телевизионных и сканерных изображений, обусловленный ограниченностью диапазона воспроизводимых яркостей. Под контрастом обычно понимают разность максимального и минимального значений яркости. Увеличение диапазона интенсивности изображения существенно улучшает качество изображения за счет усиления контраста. Для этого используют три метода – линейная растяжка, нормализация и выравнивание гистограмм. Все эти методы направлены на расширение диапазона интенсивности до максимального интервала от 0 до 255.
\newpage

\section*{Постановка задачи}
\addcontentsline{toc}{section}{Постановка задачи}
\par Цель работы – разработка последовательной и параллельных реализаций алгоритма повышения контраста полутонового изображения посредством линейной растяжки гистограммы.
\par Линейная растяжка сводится к присваиванию новых значений интенсивности каждому пикселю изображения. Если интенсивности исходного изображения изменялись в диапазоне от $i_{1}$ до $i_{2}$, тогда необходимо линейно "растянуть" указанный диапазон так, чтобы значения изменялись от 0 до 255. Для этого достаточно пересчитать старые значения интенсивности $f_{xy}$ для всех пикселей $(x,y)$ согласно формуле $g_{xy} = a \cdot f_{xy} + b$, где коэффициенты a,b просто вычисляются, исходя из того, что граница $i_{1}$ должна перейти в 0, а $i_{2}$ – в 255.
\par Нужно реализовать последовательную версию программы, а так сделать параллельную реализацию используя TBB и OpenMP.
\par Для проверки корректности работы алгоритмов требуется использовать Google C++ Testing Framework.
\newpage

\section*{Описание алгоритма}
\addcontentsline{toc}{section}{Описание алгоритма}
Алгоритм повышения контраста полутонового изображения посредством линейной растяжки гистограммы заключается в следующем:
\begin{enumerate}
\item Найти $min$ и $max$.
\item Пересчитать старые значения интенсивности для всех пикселей по формуле $f^{-1}(y)$ = (y - $y_{min}$) * ${\frac{255}{y_{max} - y_{min}}}$.
\end{enumerate}
\newpage

\section*{Описание схемы распараллеливания}
\addcontentsline{toc}{section}{Описание схемы распараллеливания}

\newpage

\section*{Описание программной реализации}
\addcontentsline{toc}{section}{Описание программной реализации}\

\newpage

\section*{Результаты экспериментов}
\addcontentsline{toc}{section}{Результаты экспериментов}
Вычислительные эксперименты проводились на оборудовании со следующей аппаратной конфигурацией:
\begin{itemize}
\item Процессор: 
\item Оперативная память: 
\item ОС: 
\end{itemize}

\newpage

\section*{Заключение}
\addcontentsline{toc}{section}{Заключение}

\newpage

\begin{thebibliography}{1}
\addcontentsline{toc}{section}{Литература}
\bibitem{Gorodetsky} Турлапов В. Е. «Обработка изображений. Часть 1».
\\URL:\url {http://www.graph.unn.ru/rus/materials/CG/CG03_ImageProcessing.pdf} (дата обращения: 22.05.2021)
\bibitem{} «Параллельное программирование на OpenMP»
\\URL:\url {http://ccfit.nsu.ru/arom/data/openmp.pdf} (дата обращения: 22.05.2021)
\bibitem{Sidnev} Сиднев А.А., Мееров И.Б., Сысоев А.В. «Разработка параллельных программ в системах с общей памятью с использованием библиотеки Intel Threading Building Blocks (TBB)».
\\URL:\url {http://hpc-education.ru/files/lectures/meerov/ppt06.pdf} (дата обращения: 22.05.2021)
\bibitem{} «Параллельное программирование. С++. Thread Support Library. Atomic Operations Library.»
\\URL:\url {https://ppt-online.org/184002} (дата обращения: 22.05.2021)
\end{thebibliography}
\newpage

\section*{Приложение}
\addcontentsline{toc}{section}{Приложение}
В данном разделе находится листинг всего кода, написанного в рамках лабораторной работы.


\end{document}